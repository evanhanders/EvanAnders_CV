\documentclass[aasms,12pt]{article}
\usepackage{natbib}
\setlength{\bibsep}{0pt plus 0.3ex}
\usepackage[margin=1in]{geometry}
\usepackage{sectsty}
\usepackage{graphicx}
\usepackage{hyperref}
\usepackage{epstopdf}
\usepackage[skip=2pt,font=small]{caption}
\captionsetup{width=\textwidth}
\usepackage{amssymb, amsmath, amsfonts, xcolor}
\hypersetup{
    colorlinks,
    linkcolor={red!50!black},
    citecolor={blue!80!black},
    urlcolor={blue!80!black}
}


\sectionfont{\normalsize}
\subsectionfont{\small}


%\citestyle{aa}
\newcommand{\sol}{\ensuremath{\odot}}
\newcommand{\RB}{Rayleigh-B\'{e}nard }
\newcommand{\grad}{\ensuremath{\nabla}}

\usepackage{fancyhdr}
\pagestyle{fancy}
\fancyhf{} % sets both header and footer to nothing
\renewcommand{\headrulewidth}{0pt}
\cfoot{\footnotesize{\thepage}}





\begin{document}
\begin{center}
   \large\textbf{Constraining the influences of rotation on stellar envelope convection at the smallest and largest scales}\\
   \vspace{0.2cm}
   \large{Evan H. Anders}\\
   \vspace{0.2cm}
\end{center}

\vspace{-0.6cm}

\section{Background \& Motivation}
The recent advent of exoplanetary science has had a fortuitous side effect: in addition to the thousands of exoplanets that have now been discovered and characterized, there is also a wealth of data on uncountable stars in our galaxy.
The long lightcurves required to detect exoplanets through the transit method necessarily contain information about the surface of potential host stars.
These large datasets have enabled the accurate determination of the radii and masses of solar-like stars on the lower main sequence with convective envelopes.
In addition to these relatively ``easy'' measurements, these asteroseismic techniques allow for the determination of (look up the other stuff that can be determined by asteroseismology).
Furthermore, this is just the beginning for asteroseismic science: datasets from current and previous missions such as Kepler and TESS have provided and will continue to provide high quality lightcurves for nearby G, K, and M type stars, and soon we will have lightcurves on hand for half a million stars along the lower main sequence.

While some asteroseismic measurements are relatively simple to perform, many stellar measurements enabled by asteroseismology depend on approximate knowledge of the interior structure of the star being studied.
The state-of-the-art method for obtaining this data is the use of one-dimensional models of stellar interiors from codes like MESA \citep{paxton&all2011}.
These models necessarily depend on one-dimensional parameterizations of convection and often employ the decades-old mixing length theory (MLT).
MLT does an adequate, but not perfect, job of describing convection within stars, and there are some well-known instances in which it regularly fails.
For example, 1D stellar models do not correctly produce the surface layers of stars with convective envelopes, but fare better when coupled with three-dimensional globa simulations \citep{jorgensen&weiss2019}.

While we are beginning to gain observational insight into stars outside of our solar system, the Sun is our closest star, and helioseismic measurements have revealed a great deal about its interior structure.
For example, the Sun's differential rotation profile is well known, including a near-surface shear layer and a ``tachocline'' separating the differentially rotating convection zone from the solid-body rotation of the radiative interior.
Along with some of these large-scale, established flows in the Sun, helioseismology has recently made us question our fundamental understanding of the nature of solar convection.
Observations by \citet{hanasoge&all2012} place an upper limit on convective velocity magnitudes nearly two orders of magnitude lower than those predicted by simulations or MLT.
Other heliseismic measurements using different techniques claim detections of velocity magnitudes more in line with what is expected from theory \citep{greer&all2015}, but observe an unexpected decrease in powers at large scales compared to simulations.
Even observations of convective motions at the solar surface \citep{hathaway&all2015} do not observe these anticipated, large-scale ``giant cells.''
Together, these observations present the Solar Convective Conundrum, and much theoretical work over the past decade has aimed to grasp what piece, or pieces, of physics long-standing models like MLT are missing.

One hypothesis is that giant cells are not driven in the deep convective zone as we expect.
Rather, cold downflows from the surface carry the solar luminosity through the whole convection zone.
This ``entropy rain'' hypothesis was first suggested by \citet{spruit1997}, and more recently has been preliminarily incorporated into a modified mixing length theory by \citet{brandenburg2016}.
Some simple simulations of \citet{kapyla&all2017} suggest that indeeed downflows may be more important than upflows in determining convective properties in envelope convection.
Recently, I conducted high resolution simulations of one possible dynamical manifestation of entropy rain: atmospheric thermals in stratified domains \citep{andersLB2019}.
To our surprise, we found that such a mechanism could plausibly carry the solar luminosity in the deep solar convection zone through their enthalpy fluxes.
However, even if entropy rain could plausibly carry the solar luminosity, it remains unclear how else these entropy rain droplets could affect the interior structure of Sun-like stars.
Specifically, these droplets would certainly carry angular momentum from the solar surface, and the location and manner in which this angular momentum is deposited could have huge implications for flow structures within the Sun, or could render the entropy rain hypothesis null.

Entropy rain is one potential mechanism for carrying the solar luminosity, but it has recently been suggested that the effects of rotation alone may be sufficient to explain the solar convective conundrum \citep{featherstone&hindman2016}.
These simulations show that as the degree of rotational constraint, characterized by the Rossby number, increases, so too does the spherical harmonic degree of maximum velocity power of the convection.
In other words, more rotationally constrained convection suppresses larger scale motions that we would otherwise predict to be driven.
While solar surface convection is not rotationally constrained, if the deep interior is rotationally constrained and at a low Rossby number then this is a plausible mechanism for the unexpectedly small amount of power at giant cell scales.
This hypothesis also has a difficulty: simulations of increasingly rotationally constrained convection typically show \emph{anti-solar} differential rotation, rather than solar-like.
Global simulations which both suppress giant cells and correctly produce the solar differential rotation profile have not been achieved.

Over the course of my postdoctoral studies, I will examine the importance of rotation on stellar convection and stellar structure from the smallest to the largest scales.
The small scale (thermal, or plume) experiments, described below in section \ref{sct:thermals}, will aim to continue to verify whether or not entropy rain is a feasible mechanism for transporting stellar luminosity in lower main sequence, solar-like stars.
The largest scale (global spherical simulations) experiments, described below in section \ref{sct:global_models} will simultaneously develop community tools to use in studying rotating convection, and also apply those tools to simulate increasingly rotationally constrained, turbulent simulations.
I will use the Dedalus pseudospectral framework \citep{burns&all2019}, which I have become proficient at using during my graduate career, to carry out these investigations.


\section{Proposed project 1: Transport of angular momentum by Entropy Rain}
\label{sct:thermals}
In order to help determine the plausibility of the entropy rain hypothesis, I recently studied the entrainment of cold atmospheric thermals in stratified domains \citep{andersLB2019}.
These thermals start off as cold entropy perturbations, quickly develop into antibuoyant vortex rings, and propagate downwards, and could be one physical manifestation of entropy rain.
In this work, I developed a theoretical description of the evolution of these vortex rings and showed that they experience a behavioral change between the unstratified (boussinesq) limit, and the highly stratified limit like that in the solar convection zone.
I conducted high resolution simulations of these thermals and showed that their evolution agreed well with our developed theory.
I also demonstrated that such a mechanism, in moving cold fluid from the solar surface downwards, could very feasibly carry the solar luminosity through the scales that we would otherwise expect giant cells to.

One huge benefit of studying small, individual dynamical phenomena like these thermals is that the resolved flows are more much turbulent than those achievable in global simulations.
The thermal simulations in \citet{andersLB2019} had a characteristic Reynolds number of approximately 600 \emph{on the length scale of the thermal}.
We found that the dynamics of these thermals were laminar, but that viscous and thermal diffusion were unimportant in the soluation.
Furthermore, at lower reynolds numbers (e.g., 100-300), we found that viscosity and diffusion became significant and greatly changed the manner in which the thermals evolved.
We are currently working on studying turbulent thermals with reynolds numbers of O(6000), as studied in \citet{lecoanet&jeevanjee2019}.
While global simulations provide great insight into the manner in which a fully convective domains self-consistently establishes broad scale structures, the individual flows achievable in global simulations naturally feel dissipation more strongly than these simulations. 
Global simulations may fit tens or hundreds of plume structures in their full domain, and while they may achieve the same domain-wide reynolds numbers, the reynolds numbers on the scale of the individual flows are naturally lower than those that we can achieve here.
As a result, these simulations give us insight into dynamics on small scales that cannot be achieved in large scale simulations.

The simulations conducted in \citet{andersLB2019} were in nonrotating domains, and I propose extending these studies into rotating domains.
During my doctoral work, I studied rapidly rotating convection in f-plane cartesian atmospheres \citep{anders&all2019}, and I propose to study thermals in similar domains.
While the thermal simulations I studied previously had two fundamental control parameters (the degree of dissipation or turbulence, characterized by the Reynolds number, and the degree of stratification), f-plane atmospheres add two additional constraints: latitude, and importance of rotation (characterized by a Rossby number).
Our past work studied entrainment and scalar transport of energy by individual thermals, and in this work we will extend that work to study vector transport of angular momentum.

I will first study an initial suite of simulations in the laminar regime in order to verify how the manner in which entropy rain deposits angular momentum as a function of both stellar latitude and the Rossby number.
After characterizing the behavioral regimes of thermals as a function of latitude and rotational constraint, I will target each of these behavioral regimes with smaller suites of turbulent simulations.
These simulations will allow us to determine whether or not the angular momentum profile of, for example, the Sun could be maintained by entropy rain.
Furthermore, we will find whether certain degrees of rotational constraint or latitudes prevent entropy rain from transiting convective envelopes.
These results could be used to create a modified version of the entropy-rain-based MLT of \citet{brandenburg2016} which includes rotation as an input parameter, which could be straightforwardly incorporated into and tested in MESA to see if it more faithfully produces the measured solar profile.


\section{Proposed project 2: Accelerated Evolution of Angular Momentum profiles}
\label{sct:global_models}
While experiments such as those presented in the previous section may give insight into convection at the smallest scales, it is important to also understand how convection behaves in a three-dimensional, time-dependent sense.
Studies of global dynamo simulations for example have shown that stars can self-consistently generate wreaths of magnetism and undergo magnetic polarity changes such as those observed in the solar dynamo.
These studies have also presented perplexing results, specifically where the Sun's differential rotation profile is concerned.
The Sun has a fast-rotating equator (rotates every 25 days) and slow rotating poles (rotates every 35 days), and deep convection is believed to be rotationally constrained.
However, when simulations which aim to model the Sun examine rotationally constrained regimes, they find antisolar differential rotation (fast poles, slow equator).
Yet, a rotationally constrained interior may be a solution to the suppression of giant cells in the Sun \citep{featherstone&hindman2016}.
These two results are counter to one another: the Sun may need to be rotationally constrained given the flows that we see at the surface, but if it is rotationally constrained, then how did it establish its differential rotation profile?

One problem that inhibits this question from being solved is that global simulations, particularly in any sort of interesting parameter regime, are expensive.
Simulations of convective dynamics inherently have relaxation timescales \citep{anders&all2018}, and the addition of rotational effects to a convective simulation means that the simulator must wait for the angular momentum profile to establish itself.
The timescale required for systems to reach thermal relaxation and for their angular momentum profiles to establish themselves limits the realm of parameter space in which converged simulations can be achieved.

During my graduate career, I studied a mechanism for accelerating the long thermal relaxation timescale in convective systems.
This mechanism is published for Boussinesq convection \citep{anders&all2018} and has been adapated to stratified, compressible convection in plane-parallel systems within our research group to be used in forthcoming work.
At even modest values of the Rayleigh number, we found that we could reach a converged state using an order of magnitude fewer computational resources than when waiting for a standard thermal relaxation timescale.
These speedups make achieving thermal relaxation in moderately turbulent systems much less computationally expensive and also allow us to study converged dynamics in more turbulent, stellar-like simulations than would otherwise be achievable on modern computing resources.

During my postdoctoral studies, I propose extending this method to global simulations in spherical domains.
This extension will involve two steps: first, I will implement a method for achieving thermal relaxation rapidly, as I have done previously.
Then, I will build upon that previous work and implement a routine which rapidly relaxes the angular momentum profile of a global simulation.
As I did in \citet{anders&all2018}, I will verify that this method produces the same results as a long relaxation in modest parameter regimes to build trust in it at the extreme parameter regimes that are interesting and otherwise unfeasible to explore.
Once developed, this tool will of course be made publicly available.

After developing this tool, I will use it to try to determine if global simulations can be produced which both suppress giant cells and properly create the solar differential rotation profile.
The overturn timescale predicted by MLT for deep convection and the solar rotational timescale are both approximately one month, and so it is possible that the Sun is in an interesting transitional regime between nonrotating and rotationally constrained convection.
I will use the tools I develop here to densely sample parameter space in this regime in order to verify whether or not solar-like convection can be observed.
In my graduate studies I determined that the degree of rotational constraint in convective simulations in cartesian domains can be straightforwardly specified \citep{anders&all2019}, so this should be quite achievable.

In addition to helping solve the convective conundrum, these accelerated evolution tools could have great benefits for asteroseismic research.
New research is beginning to couple three-dimensional, global simulations with 1-dimensional stellar structure tools in order to more accurately produce stellar structure profiles \citep{jorgensen&weiss2019}.
The fast equilibration of angular momentum and thermal profiles described here is essentially equivalent to taking timesteps which superstep the convective motions, similar to those taken in any one-dimensional stellar structure model.
These techniques presented here would therefore allow the coupling of 1-dimensional stellar structure codes with realistic statistics from three-dimensional convection to create converged stellar profiles.
Since stellar rotation times are relatively easy to measure from lightcurve data, asteroseismic inversions could take advantage of stellar structure profiles specifically evolved using three-dimensional convection and the proper degree of rotational constraint once these tools are developed.


\section{Justification of institute}
Need to do this for each application.

\bibliographystyle{apj}
\bibliography{biblio}
\end{document}
