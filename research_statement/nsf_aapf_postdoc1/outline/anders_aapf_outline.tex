\documentclass[aasms,12pt]{article}
\usepackage{natbib}
\setlength{\bibsep}{0pt plus 0.3ex}
\usepackage[margin=1in]{geometry}
\usepackage{sectsty}
\usepackage{graphicx}
\usepackage{hyperref}
\usepackage{epstopdf}
\usepackage[skip=2pt,font=small]{caption}
\captionsetup{width=\textwidth}
\usepackage{amssymb, amsmath, amsfonts, xcolor}
\hypersetup{
    colorlinks,
    linkcolor={red!50!black},
    citecolor={blue!80!black},
    urlcolor={blue!80!black}
}


\sectionfont{\normalsize}
\subsectionfont{\small}


%\citestyle{aa}
\newcommand{\sol}{\ensuremath{\odot}}
\newcommand{\RB}{Rayleigh-B\'{e}nard }
\newcommand{\grad}{\ensuremath{\nabla}}

\usepackage{fancyhdr}
\pagestyle{fancy}
\fancyhf{} % sets both header and footer to nothing
\renewcommand{\headrulewidth}{0pt}
\cfoot{\footnotesize{\thepage}}





\begin{document}
\begin{center}
   \large\textbf{AAPF outline}\\
   \vspace{0.2cm}
\end{center}

\vspace{-0.6cm}

\section{The need for improved modeling of rotation and magnetism on solar and stellar convection}
\subsection{Understanding stellar structure in the asteroseismic age}

\begin{itemize}
\item The \emph{Kepler} age has brought with it tons of data on stars, and a great byproduct of the search for exoplanets is a better understanding of stellar interiors through asteroseismology.
\item Asteroseismology is really good at getting some measurements right, like mass and radius of solar-like stars. (and describe what else it does well)
\item Asteroseismology is dependent on stellar structure models, but those parameterize convection, and we are learning that we really don't understand stellar convection...
\end{itemize}

\subsection{The Solar Convective Conundrum}

\begin{itemize}
\item Helioseismic measurements don't line up with expectations from simulations in two ways: magnitude (generally) and magnitude at large scales (giant cells).
\item Two hypotheses for the lack of giant cells are entropy rain and a rotationally constrained deep CZ.
\item We've investigated entropy rain a bit and it seems to actually be plausible.
\item However, a simpler explanation is that the interior is rotationally constrained. Although this also runs into other problems where e.g., rotationally constrained sims get anti-solar differential rotation.
\item Oh and these effects exist in a magnetized environment, too.
\end{itemize}

\subsection{The need for an understanding of solar and stellar convection.}

\begin{itemize}
\item Standard ``we need to understand the dynamo because it can mess up our technological society.''
\item Also, we need to understand convection in the context of our Sun to best use the wealth of asteroseismic data becoming available continually.
\item So here I propose to do some basic studies into the nature of how vector fields (rotation and magnetism) affect stellar convection and structure.
\end{itemize}

\section{Vector transport by convection at the smallest scales}
\begin{itemize}
\item Understanding how the smallest scales of convection transport vectors can help us to constrain which parts of physics are important across parameter space.
\item By simulating smaller scales we can achieve higher resolution, less laminar flows which more accurately reflect the physics happening in the Sun and stars.
\item We understand how convection transports \emph{scalars} at small scales, even in stratified, solar-like environments.
\item Vector transport is more complex, and my first studies will try to constrain how angular momentum and magnetic field are transported by convection at small scales (thermals and boxes).
\end{itemize}

\subsection{Task A: Vector transport by individual thermals}
\begin{itemize}
\item Thermals are regions of buoyant atmospheric fluid which fall and become buoyant vortex rings.
\item The parameter space of these simulations is: Reynolds number, Prandtl number, and stratification (and Mach number).
\item Adding more physics (magnetism, rotation) adds more dimensions to parameter space, so we will collapse this part of parameter space into: Pr = 1, high stratification, and laminar or turbulent.
\item This will allow us to explor how angular momentum and magnetism are transported in terms of parameter spaces which are important to those phenomena.
\end{itemize}

\subsubsection{Task A.1: Transport of angular momentum by thermals}
\begin{itemize}
\item The new parameters here are Rossby number and latitude.
\item We'll figure out behavior in different latitude regimes (0, 45, 90)$^{\circ}$.
\item We'll figure out behavior in rotationally constrained (low Ro) and unconstrained regimes (high Ro).
\item All this will be done in laminar studies, but we'll also study select simulations turbulently.
\end{itemize}

\subsubsection{Task A.2: Transport of magnetic fields by thermals}
\begin{itemize}
\item The new parameters here are something like a Chandrasekhar number, magnetic Pr, and initial distribution of magnetic field.
\item We'll study two magnetic field configurations (strong vertical field and a sheet of field).
\item We'll study magnetic Pr = 1 and magnetic Pr = 0.1 or 10 (whichever regime stars are in, I forget).
\item We'll study the regime of magnetism is important, and magnetism is unimportant.
\item All this will be done in laminar studies, but we'll also study select simulations turbulently.
\end{itemize}


\subsection{Task B: Vector transport by time-evolving convection in plane-parallel atmospheres}
\begin{itemize}
\item Thermals are great because they're single discrete events, but convection is an ongiong continuous process and its time-averaged statistics and evolution are important to characterize; so we'll complicate things a small amount by doing convection in a box.
\item The interactions between unstable and stable layers is interesting, in particular how vector transport acts at the interface.
\item The interface can be characterized by a stiffness, and we'll study high and low stiffness.
\item These simulations will encompass a similar parameter space to the thermals (narrowed down by those results), and stiff and squishy interfaces.
\end{itemize}


\section{Accelerated Vector Transport in Global Simulations}
\label{sct:global_models}

\begin{itemize}
\item The work done on small scales will inform a suite of large-scale models in the regimes where magnetism and/or rotation are important.
\item Stars are 3D so we need to understand convection in these geometries.
\item We can use these 3D sims to better describe \& constrain 1D models
\end{itemize}

\subsection{Building the tools for accelerated global simulations}

\begin{itemize}
\item We know how to accelerate thermal evolution, and this can even be done through proper choice of boundary conditions.
\item However large scale mean flows driven by convection often take a long time to spin up and this is essentially wasted time.
\item Using the knowledge gained in small scale simulations and the results of large scale simulations, we will build a tool to accelerate  this spin-up.
\item We will verify that this tool works in a moderate parameter regime.
\item We will use this tool to push to previously unachievable regimes.
\end{itemize}

\subsection{Linking 3D tools to stellar structure evolution}

\begin{itemize}
\item Preliminary work linking stellar structure models and 3D sims has been done to great effect.
\item These accelerated atmosphere tools are equivalent to taking large timesteps which superstep convective dynamics, just like stellar structure timesteps.
\item If time permits, we will link global simulations with 1D MESA models in order to take large timesteps in MESA with actual convective dynamics that know about rotation and/or magnetism.
\end{itemize}


\section{Collaborative studies at CIERA and Northwestern University}
\label{sct:northwestern}

\begin{itemize}
\item Description of why this will be a good place.
\end{itemize}


\section{Teaching and Outreach}
\label{sct:outreach}

\begin{itemize}
\item Outreach plans I have and how they synergize.
\end{itemize}


\section{Summary and Perspectives}
\label{sct:summary}

\end{document}
