\documentclass[11pt, preprint]{aastex62}

%%%%%%begin preamble
\usepackage[hmargin=1in, vmargin=1in]{geometry} % Margins
\usepackage{hyperref}
\usepackage{url}
\usepackage{times}
\usepackage{natbib}
\usepackage{graphicx}
\usepackage{amsmath}
\usepackage{amsfonts}
\usepackage{amssymb}
\usepackage{import}
%\usepackage{fontspec}
%\setmainfont{TimesNewRoman}
%\definetypeface[Serapion][rm][Xserif][Serapion Pro]
%\setupbodyfont[Serapion, 12pt]

%%%
%%%%%% uncomment following 4 lines to adjust title size/shape and
%%%%%% trailing space
%% \usepackage{titling}
%% %\pretitle{\noindent\Large\bfseries}
%% \date{}
%% \setlength{\droptitle}{-1in}
%\posttitle{\\}

\hypersetup{
     colorlinks   = true,
     citecolor    = gray,
     urlcolor      = blue
}

%% headers
\usepackage{fancyhdr}
\pagestyle{fancy}
\lhead{Anders}
\chead{}
\rhead{Title goes here}
\lfoot{}
\cfoot{\thepage}
\rfoot{}

\newcommand{\nosection}[1]{%
  \refstepcounter{section}%
  \addcontentsline{toc}{section}{\protect\numberline{\thesection}#1}%
  \markright{#1}}
\newcommand{\nosubsection}[1]{%
  \refstepcounter{subsection}%
  \addcontentsline{toc}{subsection}{\protect\numberline{\thesubsection}#1}%
  \markright{#1}}

\usepackage{titlesec}
%\titleformat{\abstract}[runin]{\normalfont\normalsize\bfseries}{\theabstract}{1em}{}
\titleformat{\section}[block]{\normalfont\fontsize{14pt}{17pt}\selectfont\bfseries\filcenter}{\thesection}{1em}{}
\titleformat{\subsection}[runin]{\normalfont\fontsize{12.5pt}{14pt}\selectfont\bfseries}{\thesubsection}{1em}{}
\titleformat{\subsubsection}[runin]{\normalfont\normalsize\bfseries}{\thesubsubsection}{1em}{}




%%%%%%end preamble

\begin{document}
\thispagestyle{empty}
\parindent=0cm
\begin{center}
\huge{Evan H. Anders}\\
\large{Biographical Sketch}\\
\small{Department of Astrophysical and Planetary Sciences\\
University of Colorado, Boulder \\\texttt{evan.anders@colorado.edu}}
\end{center}


\section*{Citizenship}
\vspace{-0.15cm}
\begin{center}
Evan H. Anders is a citizen of the United States of America.
\end{center}

\section*{Professional Preparation}
\vspace{-0.15cm}
\begin{tabular}{lll}
  Whitworth University & Physics & BS May 2014\\
  University of Colorado, Boulder & Astrophysics & MS December 2017\\
  University of Colorado, Boulder & Astrophysics & PhD Expected May 2020\\
\end{tabular}

\section*{Appointments}
\vspace{-0.15cm}
\begin{tabular}{lll}
  NASA NESSF/FINESST Fellow & University of Colorado, Boulder & Sept 2018 -- \\
  George Ellery Hale Fellow & University of Colorado, Boulder & Sept 2015 -- Aug 2018 \\
\end{tabular}

\section*{Products}
\vspace{-0.15cm}
\begin{itemize}
\setlength{\itemsep}{-\parsep}
\setlength{\topsep}{-2\parsep}
\setlength{\partopsep}{-2\parsep}

\item \textbf{Anders}, E.~H., Lecoanet, D., \& Brown, B.~P.,
``Entropy Rain: Dilution and Compression of Thermals in Stratified Domains'',
2019, \href{https://ui.adsabs.harvard.edu/abs/2019arXiv190602342A/abstract}{Accepted for publication in \emph{The Astrophysical Journal}}

\item \textbf{Anders}, E.~H., Manduca, C.~M., Brown, B.~P., Oishi, J.~S., \& Vasil, G.~M.,
``Predicting the Rossby Number in Convective experiments'',
2019, \href{https://iopscience.iop.org/article/10.3847/1538-4357/aaff61}{\emph{The Astrophysical Journal} 872,~2.}

\item \textbf{Anders}, E.~H., Brown, B.~P., \& Oishi, J.~S.,
``Accelerated evolution of convective simulations'',
2018, \href{https://journals.aps.org/prfluids/abstract/10.1103/PhysRevFluids.3.083502}{\emph{Physical Review Fluids} 3,~8:083502.}

\item \textbf{Anders}, E.~H. \& Brown, B.~P.,
``Convective heat transport in stratified atmospheres at low and high Mach number'',
2018, \href{https://journals.aps.org/prfluids/abstract/10.1103/PhysRevFluids.2.083501}{\emph{Physical Review Fluids} 2,~8:083501.}

\end{itemize}


\section*{Synergistic activities}
\vspace{-0.15cm}
\begin{itemize}
\setlength{\itemsep}{-\parsep}
\setlength{\topsep}{-2\parsep}
\setlength{\partopsep}{-2\parsep}

\item Anders has designed, tested, and made public a method of simulating state-of-the-art convection simulations which requires an order of magnitude fewer computational hours than traditional methods.
\item Anders has served multiple years each on CU Boulder's graduate admissions committee, graduate exams committee, and faculty search committees; he has led the development and use of rubrics in the graduate admissions process in his department which ensure more equitable evaluation of applicants.
\item Anders spent three years serving as an administrator of the CU-Boulder Science, Technology, and Astronomy RecruitS (CU-STARS) program, with duties including coordinating outreach trips to high schools, mentoring undergraduates, and designing hands-on activities in exoplanetary science, black holes, and atmospheric dynamics.
\item Anders has helped mentor two undergraduate students in year-long projects; one of these projects eventually resulted in a publication (Anders, Manduca et al. 2019). 

\end{itemize}

\end{document}
