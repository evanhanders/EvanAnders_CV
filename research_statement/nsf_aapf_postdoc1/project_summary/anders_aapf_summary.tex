\documentclass[aasms,11pt]{article}
\usepackage{natbib}
\setlength{\bibsep}{0pt plus 0.3ex}
\usepackage[margin=1in]{geometry}
\usepackage{sectsty}
\usepackage{graphicx}
\usepackage{hyperref}
\usepackage{epstopdf}
\usepackage{anyfontsize}
\usepackage{wrapfig}
\usepackage[skip=2pt,font=small]{caption}
\captionsetup{width=\textwidth}
\usepackage{amssymb, amsmath, amsfonts, xcolor}
\hypersetup{
    colorlinks,
    linkcolor={red!50!black},
    citecolor={blue!80!black},
    urlcolor={blue!80!black}
}


\sectionfont{\normalsize}
\subsectionfont{\small}


%\citestyle{aa}
\newcommand{\aj}{The Astronomical Journal}
\newcommand{\apj}{The Astrophysical Journal}
\newcommand{\apjl}{The Astrophysical Journal Letters}
\newcommand{\apjs}{The Astrophysical Journal Supplemental Series}
\newcommand{\aap}{Astronomy \& Astrophysics}
\newcommand{\aaps}{Astronomy \& Astrophysics Supplemental Series}
\newcommand{\mnras}{Monthly Notices of the Royal Astronomical Society}
\newcommand{\baas}{Bulletin of the American Astronomical Society}
\newcommand{\zap}{Zeitschrift für Astrophysik}
\newcommand{\sol}{\ensuremath{\odot}}
\newcommand{\RB}{Rayleigh-B\'{e}nard }
\newcommand{\grad}{\ensuremath{\nabla}}

\usepackage{fancyhdr}
\pagestyle{fancy}
\fancyhf{} % sets both header and footer to nothing
\renewcommand{\headrulewidth}{0pt}
\renewcommand{\footrulewidth}{0pt}
\rfoot{\footnotesize{Evan H. Anders NSF AAPF 2019}}
\lfoot{\footnotesize{\leftmark}}
%\lfoot{\footnotesize{\leftmark}}%\thesection: \sectiontitle}}
\cfoot{\footnotesize{\thepage}}

\makeatletter
\renewcommand{\sectionmark}[1]{%
  \markboth{\ifnum \c@secnumdepth>\z@
      \thesection: \hskip 1em\relax
    \fi #1}{}}
\makeatother

\usepackage{titlesec}
%\titleformat{\abstract}[runin]{\normalfont\normalsize\bfseries}{\theabstract}{1em}{}
\titleformat{\section}[runin]{\normalfont\fontsize{11pt}{17pt}\selectfont\bfseries}{\thesection}{1em}{}
\titleformat{\subsection}[runin]{\normalfont\fontsize{12.5pt}{14pt}\selectfont\bfseries}{\thesubsection}{1em}{}
\titleformat{\subsubsection}[runin]{\normalfont\normalsize\bfseries}{\thesubsubsection}{1em}{}

\renewenvironment{abstract}
{
\noindent \par{\bfseries \abstractname.}}
{\medskip\noindent 
}





\begin{document}

\section{Overview}
\vspace{-11pt}
Asteroseismic observations of stellar pulsations are now plentiful, but these observations often depend on stellar structure models which employ simple parameterizations of convection. 
These parameterizations such as Mixing Length Theory fail to accurately describe the complexities of stellar convection.
Giant convective cells are absent in observations of the Sun despite being predicted by theory, simulations, and these parameterizations; their absence is referred to as the Solar Convective Conundrum.
Finding a solution to this conundrum is crucial for improving asteroseismic measurements and gaining an understanding of the Sun's magnetic dynamo.

Modern numerical simulations are often very complex, and this makes it hard to understand why they disagree with observations.
Historically, simpler simulations were conducted where basic physical processes could be explored, but these have not been prevalent in recent years and the state-of-the-art is often two decades old.
Here I propose three studies into solar and stellar convection which utilize both simple and complex systems to examine convection from the smallest to largest spatial scales.
In task A, I will simulate convection at the smallest scales and learn how magnetism and rotation affect intense stellar downflows. 
In task B, I will simulate mesoscale convection at the radiative-convective boundary deep inside of solar-type stars to learn about dynamo pumping mechanisms across that interface. 
In task C, I will develop a community tool to reduce the computational cost of global spherical simulations, and I will utilize this tool to simulate and study dynamo processes in properly equilibrated simulations.

In addition to these research studies, I will develop a partnership between graduate students and local teachers in order to improve STEM educator training and education outcomes at the secondary and post-secondary levels, as described below in the Broader Impacts section. 
The proposed host institution for these studies is the Center for Interdisciplinary Exploration and Research in Astrophysics (CIERA) at Northwestern University, and the sponsoring scientist is Prof. Daniel Lecoanet.

\vspace{-7pt}
\section{Intellectural Merit}
\vspace{-11pt}
The simulations proposed are necssary and timely. 
Current convective models cannot explain the Convective Conundrum and the combined effects of downflows, rotation, and magnetism is unclear and has not been well explored. 
The stellar structure models which this work will inform are an integral piece of asteroseismic measurements; these measurements are being used not only to examine stellar interiors but also to determine the sizes of exoplanets and also in galactic archaeology.
Some modern studies are coupling three-dimensional convective simulations with one-dimensional stellar structure models, and the tool developed in task C will allow such a method to be employed for generalized stars.
Simulations-as-observations are being widely used to understand solar surface convection and to prepare for first light on the NSF-supported DKIST telescope, and this work will determine why some aspects of those simulations disagree with the Sun to improve and build confidence in these simulations moving forward.


\vspace{-7pt}
\section{Broader Impacts}
\vspace{-11pt}
I will design a workshop series which partners scientific experts (graduate students) with pedagogical experts (local high school teachers) with the goals of better developing STEM educators and improving STEM engagement and learning outcomes at the secondary and post-secondary levels.
I will furthermore partner with teacher organizations in the Chicago area in order to ensure participation of high school educators and to create lasting connections between these two groups of experts.
Over the course of this workshop series, scientific experts will learn the basics of teaching pedagogy and course design while pedagogical experts will expand their content knowledge within a self-identified scientific topic.
These experts will collaboratively design teaching modules for Next Generation Science Standards (NGSS) core ideas and practices which will be widely distributed to high school classrooms with the help of our partner teacher organizations.

\end{document}
