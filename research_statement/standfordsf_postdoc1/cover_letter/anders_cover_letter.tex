\documentclass[11pt, preprint]{aastex}

%%%%%%begin preamble
\usepackage[hmargin=1in, vmargin=1in]{geometry} % Margins
\usepackage{hyperref}
\usepackage{url}
\usepackage{times}
\usepackage{natbib}
\usepackage{graphicx}
\usepackage{amsmath}
\usepackage{amsfonts}
\usepackage{amssymb}
\usepackage{pdfpages}
%\usepackage{fontspec}
%\setmainfont{TimesNewRoman}
%\definetypeface[Serapion][rm][Xserif][Serapion Pro]
%\setupbodyfont[Serapion, 12pt]

%%%
%%%%%% uncomment following 4 lines to adjust title size/shape and
%%%%%% trailing space
%% \usepackage{titling}
%% %\pretitle{\noindent\Large\bfseries}
%% \date{}
%% \setlength{\droptitle}{-1in}
%\posttitle{\\}

\hypersetup{
     colorlinks   = true,
     citecolor    = gray,
     urlcolor      = blue
}

\setcounter{tocdepth}{2}
%% headers
%% \usepackage{fancyhdr}
%% \pagestyle{fancy}
%% \lhead{Bordwell, Jhett}
%% \chead{}
%% \rhead{Curriculum Vitae}
%% \lfoot{}
%% \cfoot{\thepage}
%% \rfoot{}
\newcommand{\sol}{\ensuremath{\odot}}
\newcommand{\Dedalus}{\href{http://dedalus-project.org}{Dedalus}}
\newcommand{\Reyn}{\ensuremath{\mathrm{Re}}}
\newcommand{\Rayleigh}{\ensuremath{\mathrm{Ra}}}
\newcommand{\Rossby}{\ensuremath{\mathrm{Ro}}}
\newcommand{\Rmag}{\ensuremath{\mathrm{Rm}}}
\newcommand{\Rmagc}{\ensuremath{\mathrm{Rm}_\mathrm{crit}}}
\newcommand{\Prandtl}{\ensuremath{\mathrm{Pm}}}
\newcommand{\Peclet}{\ensuremath{\mathrm{Pe}}}
\newcommand{\Mach}{\ensuremath{\mathrm{Ma}}}
\newcommand{\Stiffness}{\ensuremath{\mathrm{S}}}
\newcommand{\Lund}{\ensuremath{\mathrm{S}}}
\newcommand{\Lundc}{\ensuremath{\mathrm{S}_\mathrm{crit}}}
\newcommand{\yt}{\texttt{yt}}
\newcommand{\enzo}{\texttt{Enzo}}
\newcommand{\nosection}[1]{%
  \refstepcounter{section}%
  \addcontentsline{toc}{section}{\protect\numberline{\thesection}#1}%
  \markright{#1}}
\newcommand{\nosubsection}[1]{%
  \refstepcounter{subsection}%
  \addcontentsline{toc}{subsection}{\protect\numberline{\thesubsection}#1}%
  \markright{#1}}
\newcommand{\DOT}{\hspace{0.15cm}$\cdot$\hspace{0.15cm}}

%%%%%%end preamble

\begin{document}
\thispagestyle{empty}
\parindent=0cm
\section*{~}
\vspace{-1.5cm}
\begin{flushright}
December 19$^\text{th}$, 2018
\end{flushright}
\begin{flushleft}
  Stanford Science Fellows Program \\%\begin{flushright}\vspace{-0.4in} December 19th, 2018\vspace{-0.15in}\end{flushright}
  Stanford University \DOT
  Stanford, California 94305
\end{flushleft}
Dear Selection Committee,\vspace{-0.15in}\\\\
I am writing to apply for your postdoctoral research position in theoretical studies of exoplanet atmospheres. Along with this letter, in the following pages you will find my curriculum vitae with a list of publications (three pages) and a description of my research interests and experience (three pages). As references, I list Prof. Benjamin P. Brown (Univ. of Colorado Boulder), Prof. Jeffrey S. Oishi (Bates), and Prof. Zachory K. Berta-Thompson (Univ. of Colorado Boulder). I provide their contact information below, and they will submit letters to you separately by December 21st.

As a graduate student at the University of Colorado Boulder, I have performed several theoretical studies aimed at investigating the consequences of small-scale dynamics in Jovian atmospheres on chemical transport. I have performed both 2 and 3D local simulations of convection and convection with an overlying stably stratified region using the \emph{Dedalus} pseudospectral code to generate numerically-validated parameterizations for transport. These fully compressible simulations have been performed in cutting edge, highly turbulent regimes at high resolutions (512x1024$^2$ at a Rayleigh number of 10$^6$) in a parallel fashion on modern supercomputers such as NASA's \emph{Pleiades}. Additionally, I am performing an analytical study using multi-scale weakly nonlinear analysis to examine the evolution of small-scale gravity waves in the presence of dissipative effects (e.g., radiative damping) in a variety of Jovian archetypes.

The Department of Astronomy at the University of Michigan hosts a variety of people and resources that would benefit a research program aimed at tying together small-scale local studies to observational consequences using a 3D general circulation model. Faculty members such as Edwin Bergin and Michael Meyer would be natural collaborators for work on chemical transport and observational consequences for modern and upcoming telescopes. Members of the Department of Climate and Space Science such as Sushil Atreya would also be well-suited to applications of this work within and outside of the Solar system. Finally, your group's expertise with the Intermediate General Circulation Model would be invaluable to the above-described research program.

Thank you for considering my application,
\begin{figure}[!ht]
  \flushleft
  \includegraphics[width=2in]{evan_signature.pdf}
\end{figure}
\vspace{-0.6in}
\begin{flushleft}
  Evan H. Anders\\
  Dept. of Astrophysical and Planetary Sciences\DOT
  University of Colorado Boulder\DOT
  Boulder, CO 80309
\end{flushleft}

\begin{table}[!ht]
  \flushleft
  \begin{tabular}{lllllll}
  \textbf{REFERENCES}&&&&\\
  Benjamin P. Brown 	&&& Daniel Lecoanet 		&&& Jeffrey S. Oishi\\
  bpbrown@colorado.edu  &&& lecoanet@princeton.edu  &&& joishi@bates.edu\\
  \end{tabular}
\end{table}

\end{document}
