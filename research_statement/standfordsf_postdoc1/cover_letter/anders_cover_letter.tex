\documentclass[11pt, preprint]{aastex}

%%%%%%begin preamble
\usepackage[hmargin=1in, vmargin=1in]{geometry} % Margins
\usepackage{hyperref}
\usepackage{url}
\usepackage{times}
\usepackage{natbib}
\usepackage{graphicx}
\usepackage{amsmath}
\usepackage{amsfonts}
\usepackage{amssymb}
\usepackage{pdfpages}
\usepackage{tabularx}
%\usepackage{fontspec}
%\setmainfont{TimesNewRoman}
%\definetypeface[Serapion][rm][Xserif][Serapion Pro]
%\setupbodyfont[Serapion, 12pt]

%%%
%%%%%% uncomment following 4 lines to adjust title size/shape and
%%%%%% trailing space
%% \usepackage{titling}
%% %\pretitle{\noindent\Large\bfseries}
%% \date{}
%% \setlength{\droptitle}{-1in}
%\posttitle{\\}

\hypersetup{
     colorlinks   = true,
     citecolor    = gray,
     urlcolor      = blue
}

\setcounter{tocdepth}{2}
%% headers
%% \usepackage{fancyhdr}
%% \pagestyle{fancy}
%% \lhead{Bordwell, Jhett}
%% \chead{}
%% \rhead{Curriculum Vitae}
%% \lfoot{}
%% \cfoot{\thepage}
%% \rfoot{}
\newcommand{\sol}{\ensuremath{\odot}}
\newcommand{\Dedalus}{\href{http://dedalus-project.org}{Dedalus}}
\newcommand{\Reyn}{\ensuremath{\mathrm{Re}}}
\newcommand{\Rayleigh}{\ensuremath{\mathrm{Ra}}}
\newcommand{\Rossby}{\ensuremath{\mathrm{Ro}}}
\newcommand{\Rmag}{\ensuremath{\mathrm{Rm}}}
\newcommand{\Rmagc}{\ensuremath{\mathrm{Rm}_\mathrm{crit}}}
\newcommand{\Prandtl}{\ensuremath{\mathrm{Pm}}}
\newcommand{\Peclet}{\ensuremath{\mathrm{Pe}}}
\newcommand{\Mach}{\ensuremath{\mathrm{Ma}}}
\newcommand{\Stiffness}{\ensuremath{\mathrm{S}}}
\newcommand{\Lund}{\ensuremath{\mathrm{S}}}
\newcommand{\Lundc}{\ensuremath{\mathrm{S}_\mathrm{crit}}}
\newcommand{\yt}{\texttt{yt}}
\newcommand{\enzo}{\texttt{Enzo}}
\newcommand{\nosection}[1]{%
  \refstepcounter{section}%
  \addcontentsline{toc}{section}{\protect\numberline{\thesection}#1}%
  \markright{#1}}
\newcommand{\nosubsection}[1]{%
  \refstepcounter{subsection}%
  \addcontentsline{toc}{subsection}{\protect\numberline{\thesubsection}#1}%
  \markright{#1}}
\newcommand{\DOT}{\hspace{0.15cm}$\cdot$\hspace{0.15cm}}

%%%%%%end preamble

\begin{document}
\thispagestyle{empty}
\parindent=0cm
\section*{~}
\vspace{-1.5cm}
\begin{flushright}
October 26$^\text{th}$, 2019
\end{flushright}
\begin{flushleft}
  Stanford Science Fellows Program \\%\begin{flushright}\vspace{-0.4in} December 19th, 2018\vspace{-0.15in}\end{flushright}
  Stanford University \DOT
  Stanford, California 94305
	\vspace{-11pt}
\end{flushleft}
Dear Selection Committee,\vspace{-0.15in}\\\\
I am writing to apply for a Stanford Science Fellowship.
Along with this letter, my curriculum vitae and brief descriptions of my career goals and research problems I will study as a fellow have been uploaded to AcademicJobsOnline.

As a graduate research fellow at the University of Colorado---Boulder, I have led several theoretical studies which have investigated fundamental processes in stellar convection.
I have performed two- and three-dimensional simulations of convection in both highly simplified systems and in more complex, fully compressible, stratified atmospheres.
These studies have elucidated how to control turbulence in both rotating and nonrotating systems and have studied the nature of turbulent convection.
These studies have further led to the development of a tool which can equilibrate convective simulations using ten times fewer cpu-hours than traditional methods.
More recently, I have investigated simulations of discrete convective downflows and learned how density stratification affects their propagation.
My doctoral studies have employed the \emph{Dedalus} pseudospectral code and were performed in cutting edge, highly resolved (2048x384$^2$), and highly turbulent regimes through using massively parallel techniques on  modern supercomputers like NASA's \emph{Pleiades}.
As I wrap up my graduate studies with Prof.~Benjamin Brown (Univ.~Colorado), I am working on two collaborative research projects which build on my previous work.
First, along with Prof.~Daniel Lecoanet (Princeton / Northwestern) and Dr.~Lydia Korre (Univ.~Colorado), I am extending my studies of individual downflows to learn about their interactions with convectively stable regions, as these interactions are very important in the deep solar convection zone.
Second, I am collaborating with Prof.~Geoff Vasil (Univ.~Sydney) to show that thermal equilibration in rotating convection can cause convective flows at early simulation times to be in very different dynamical regimes from equilibrated flows.

Stanford University hosts numerous experts who I am excited to collaborate with as I extend my doctoral research to the problems I have proposed in my research statement.
Stanford's Physics department is a natural home for my research, and I will collaborate closely with Profs.~Abel, Petrosian, and Wagoner.
From the Physics department, I look forward to creating cross-disciplinary connections with experts in the Depts.~Earth System Science and Mathematics and the Center for Turbulence Research.
Beyond research, I am eager to participate in the Stanford Science Fellows program's comunity-building and professional development activities as I continue to work towards my ultimate career goal of becoming a professor.

Thank you for considering my application,
\begin{figure}[!ht]
  \vspace{-11pt}
  \flushleft
  \includegraphics[width=2in]{evan_signature.pdf}
\end{figure}
\vspace{-0.6in}
\begin{flushleft}
  Evan H.~Anders\\
  NASA Earth and Space Science Fellow\\
  Dept.~Astrophysical and Planetary Sciences\\
  University of Colorado---Boulder\\
  Boulder, CO 80309
\end{flushleft}

\end{document}
