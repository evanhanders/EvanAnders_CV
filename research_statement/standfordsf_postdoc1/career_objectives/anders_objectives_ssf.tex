\documentclass[aps, pre, onecolumn, nofootinbib, notitlepage, groupedaddress, amsfonts, amssymb, amsmath]{revtex4-1}
%%%%%%begin preamble
\usepackage[hmargin=1in, vmargin=1in]{geometry} % Margins
\usepackage{hyperref}
\usepackage{url}
\usepackage{natbib}
\setlength{\bibsep}{0pt plus 0.3ex}
\usepackage{graphicx}
%\usepackage{pdfpages} % breaks with aastex6
\usepackage{import}
\usepackage{wrapfig}

\usepackage{xcolor}
\hypersetup{
  colorlinks   = true,
  %citecolor    = blue
  citecolor    = gray
  % gray is not being found!?!
  % gray is found if pdfpages is used... crap.
  %citecolor    = grey
  %citecolor    = Gray
}

%Copy+pasted from aastex.
\newcommand\aj{\ref@jnl{AJ}}%        % Astronomical Journal 
\newcommand\araa{\ref@jnl{ARA\&A}}%  % Annual Review of Astron and Astrophys 
\renewcommand\apj{\ref@jnl{ApJ}}%    % Astrophysical Journal ++
\newcommand\apjl{\ref@jnl{ApJL}}     % Astrophysical Journal, Letters 
\newcommand\apjs{\ref@jnl{ApJS}}%    % Astrophysical Journal, Supplement 
\renewcommand\ao{\ref@jnl{ApOpt}}%   % Applied Optics ++
\newcommand\apss{\ref@jnl{Ap\&SS}}%  % Astrophysics and Space Science 
\newcommand\aap{\ref@jnl{A\&A}}%     % Astronomy and Astrophysics 
\newcommand\aapr{\ref@jnl{A\&A~Rv}}%  % Astronomy and Astrophysics Reviews 
\newcommand\aaps{\ref@jnl{A\&AS}}%    % Astronomy and Astrophysics, Supplement 
\newcommand\azh{\ref@jnl{AZh}}%       % Astronomicheskii Zhurnal 
\newcommand\baas{\ref@jnl{BAAS}}%     % Bulletin of the AAS 
\newcommand\icarus{\ref@jnl{Icarus}}% % Icarus
\newcommand\jrasc{\ref@jnl{JRASC}}%   % Journal of the RAS of Canada 
\newcommand\memras{\ref@jnl{MmRAS}}%  % Memoirs of the RAS 
\newcommand\mnras{\ref@jnl{MNRAS}}%   % Monthly Notices of the RAS 
\renewcommand\pra{\ref@jnl{PhRvA}}% % Physical Review A: General Physics ++
\renewcommand\prb{\ref@jnl{PhRvB}}% % Physical Review B: Solid State ++
\renewcommand\prc{\ref@jnl{PhRvC}}% % Physical Review C ++
\renewcommand\prd{\ref@jnl{PhRvD}}% % Physical Review D ++
\renewcommand\pre{\ref@jnl{PhRvE}}% % Physical Review E ++
\renewcommand\prl{\ref@jnl{PhRvL}}% % Physical Review Letters 
\newcommand\pasp{\ref@jnl{PASP}}%     % Publications of the ASP 
\newcommand\pasj{\ref@jnl{PASJ}}%     % Publications of the ASJ 
\newcommand\qjras{\ref@jnl{QJRAS}}%   % Quarterly Journal of the RAS 
\newcommand\skytel{\ref@jnl{S\&T}}%   % Sky and Telescope 
\newcommand\solphys{\ref@jnl{SoPh}}% % Solar Physics 
\newcommand\sovast{\ref@jnl{Soviet~Ast.}}% % Soviet Astronomy 
\newcommand\ssr{\ref@jnl{SSRv}}% % Space Science Reviews 
\newcommand\zap{\ref@jnl{ZA}}%       % Zeitschrift fuer Astrophysik 
\renewcommand\nat{\ref@jnl{Nature}}%  % Nature 
\newcommand\iaucirc{\ref@jnl{IAUC}}% % IAU Cirulars 
\newcommand\aplett{\ref@jnl{Astrophys.~Lett.}}%  % Astrophysics Letters 
\newcommand\apspr{\ref@jnl{Astrophys.~Space~Phys.~Res.}}% % Astrophysics Space Physics Research 
\newcommand\bain{\ref@jnl{BAN}}% % Bulletin Astronomical Institute of the Netherlands 
\newcommand\fcp{\ref@jnl{FCPh}}%   % Fundamental Cosmic Physics 
\newcommand\gca{\ref@jnl{GeoCoA}}% % Geochimica Cosmochimica Acta 
\newcommand\grl{\ref@jnl{Geophys.~Res.~Lett.}}%  % Geophysics Research Letters 
\renewcommand\jcp{\ref@jnl{JChPh}}%     % Journal of Chemical Physics 
\newcommand\jgr{\ref@jnl{J.~Geophys.~Res.}}%     % Journal of Geophysics Research 
\newcommand\jqsrt{\ref@jnl{JQSRT}}%   % Journal of Quantitiative Spectroscopy and Radiative Trasfer 
\newcommand\memsai{\ref@jnl{MmSAI}}% % Mem. Societa Astronomica Italiana 
\newcommand\nphysa{\ref@jnl{NuPhA}}%     % Nuclear Physics A 
\newcommand\physrep{\ref@jnl{PhR}}%       % Physics Reports 
\newcommand\physscr{\ref@jnl{PhyS}}%        % Physica Scripta 
\newcommand\planss{\ref@jnl{Planet.~Space~Sci.}}%  % Planetary Space Science 
\newcommand\procspie{\ref@jnl{Proc.~SPIE}}%      % Proceedings of the SPIE 

\newcommand\actaa{\ref@jnl{AcA}}%  % Acta Astronomica
\newcommand\caa{\ref@jnl{ChA\&A}}%  % Chinese Astronomy and Astrophysics

\setcounter{tocdepth}{2}
%% headers
\usepackage{fancyhdr}
\pagestyle{fancy}
\fancyhf{} % sets both header and footer to nothing
\lhead{Evan Anders---Research Statement}
\rhead{Stanford University, Stanford Science Fellows}
\cfoot{\footnotesize{\thepage}}
%\pagestyle{empty}
%\pagenumbering{gobble}
%\renewcommand*{\thefootnote}{\fnsymbol{footnote}}

\renewcommand{\vec}{\ensuremath{\boldsymbol}}
\newcommand{\dedalus}{\href{http://dedalus-project.org}{Dedalus}}
\newcommand{\del}{\ensuremath{\vec{\nabla}}}
\newcommand{\scrS}{\ensuremath{\mathcal{S}}}

\newcommand{\nosection}[1]{%
  \refstepcounter{section}%
  \addcontentsline{toc}{section}{\protect\numberline{\thesection}#1}%
  \markright{#1}}
\newcommand{\nosubsection}[1]{%
  \refstepcounter{subsection}%
  \addcontentsline{toc}{subsection}{\protect\numberline{\thesubsection}#1}%
  \markright{#1}}

%\usepackage{atbegshi}
%%%%%%end preamble


\begin{document}

During my PhD, I gained proficiency in using the open-source Dedalus \citep{burns&all2019} code to study questions about stellar convection at various scales.
I used Dedalus to study thermals-as-downflows and learned how atmospheric stratification affects the propagation of these downflows \citep{andersLB2019}.
I surprisingly found that these downflows carry enough energy to feasibly transport the solar luminosity, giving credence to the entropy rain hypothesis.
However, this work neglected turbulence, magnetism, and rotation, which are key ingredients in stellar convection.

In creating simulations of solar convection, it is important to strive to produce flows which feel the same influence of rotation and magnetism as those in the Sun.
Simulations which include global rotation can be run in a regime where flows are heavily affected by Coriolis forces or one where Coriolis forces are weak.
A ``critical'' rotation frequency, $\Omega_\text{c}$, separates these two regimes, but determining $\Omega_\text{c}$ is often not straightforward.
During my PhD, I found a method for determining $\Omega_\text{c}$ so that the importance of rotation could be specified in simulation initial conditions \citep[see left panel of Fig. \ref{fig:rossby_plot} and][]{anders&all2019}.
Flows look very different depending on whether you simulate above, below, or near $\Omega_\text{c}$ (right panels of Fig.~\ref{fig:rossby_plot}), and it is crucial to simulate in the same regime as the Sun.
In magnetized systems, there is an analagous critical magnetic field, but to date no study has determined how to determine this field \emph{a priori}.

During my PhD, I created and verified an ``accelerated evolution'' tool which skips the long KH timescale in convective simulations \citep{anders&all2018}.
This tool reached a relaxed, equilibrated state using an order of magnitude fewer computational resources than a simulation which timestepped through a full KH timescale, and results between the simulations differed by less than 1\%.

\bibliographystyle{yahapj}
\bibliography{biblio}
\end{document}
