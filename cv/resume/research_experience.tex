%-------------------------------------------------------------------------------
%	SECTION TITLE
%-------------------------------------------------------------------------------
\cvsection{Research Experience}


%-------------------------------------------------------------------------------
%	CONTENT
%-------------------------------------------------------------------------------
\begin{cventries}

%---------------------------------------------------------
  \cventry
    {Graduate Research Fellow / Assistant} % Job Title
    {CU Boulder \& Laboratory for Atmospheric and Space Physics (LASP)} % Organization
    {Boulder, CO} % Location
    {May 2015 - Present} % Date(s)
    {
      \begin{cvitems} % Description(s) of tasks/responsibilities
	  	\item { Published four first-author papers in the Astrophysical Journal and Physical Review Fluids }
        \item { Became proficient in creating and analyzing simulations using the Dedalus pseudospectral framework. }
      \end{cvitems}
    }

%---------------------------------------------------------
  \cventry
    {NSF SURF Fellow} % Job title
    {Laser Interferometer Gravitational-Wave Observatory (LIGO)} % Organization
    {Hanford, WA} % Location
    {Summer 2013} % Date(s)
    {
%      \begin{cvitems} % Description(s) of tasks/responsibilities
%        \item {Developed a tool in Python to analyze calibration lines in LIGO's power spectrum.}
%        \item {Analyzed the consistency between input and output channels in LIGO's photon calibration system.}
%      \end{cvitems}
    }

%---------------------------------------------------------
  \cventry
    {DOE SULI Intern} % Job title
    {Pacific Northwest National Laboratory (PNNL)} % Organizatio)n
    {Richland, WA} % Location
    {Summer 2012} % Date(s)
    {
%      \begin{cvitems} % Description(s) of tasks/responsibilities
%        \item {Optimized functions in GAiN, a Python module which applies PNNL’s Global Arrays parallel
%        programming toolkit to the NumPy Python module.}
%        \item {Designed new parallel algorithms for the GAiN `reduce' function and developed the foundation of the GAiN `master-slave' interface.}
%      \end{cvitems} 
    }


%---------------------------------------------------------
\end{cventries}
